%%This is a very basic article template.

\documentclass[a4paper,10pt]{article}
\usepackage{latexsym}
\usepackage{amssymb}
\usepackage{bm}
\usepackage{graphicx}
\usepackage{natbib}
\usepackage{geometry}
\usepackage{pdflscape}
\usepackage{hyperref}


\newgeometry{margin=2cm}
\begin{document}

\title{\flushleft\small{Department Software and Computer Technology\\ 
Faculty EEMCS\\
DELFT UNIVERSITY OF TECHNOLOGY\\}
\center {\Large{Assignment-1 \\
(Tree Pattern Evaluation using SAX)\\
Web Data Management (IN4331)\\
2012-2013}}}
\author{Ioanna Jivet() \\
Nidhi Singh (4242246)}
        
\date{\today}
    
\maketitle

\section{Basic Setup}

For this exercise, we have used two SAX parsers, one which parses the input tree pattern and the other which parses the main XML document
to look for the input tree pattern and find matches. We decided to take the input as an XML file since it is easier to parse and gives an oppotunity 
to use SAX again.

The input tree pattern should contain the tree pattern nodes as tags, attributes should be assigned to show if a node is \emph{marked}, or has an 
\emph{optional} edge between its parent.
\begin{itemize}
  \item \textbf{marked}: takes boolean value \emph{true/false}, default: false
  \item \textbf{optional}: takes boolean value \emph{true/false}, default: false
\end{itemize}
The text within each tag is treated similar to the predicate values in \emph{where} clause of a query. Matched objects are created based on this condition.

We have included sample input XML file indicating how the tree pattern XML file should look. 
\section{Exercises}
\subsection{Implementation of an evaluation algorithm for C-TP tree-patterns}
\subsection{Implementation of an algorithm that computes the result tuples of C-TP tree patterns}
\subsection{Extension to include wildcard ``*'' node}
\subsection{Extension to include optional nodes}
\subsection{support for predicate values}
\subsection{Eextension to return subtrees and not only preorder numbers}
\end{document}